\documentclass[a4paper,10pt]{article}
\usepackage[utf8]{inputenc}
\usepackage{polski}
\usepackage[polish]{babel}
\usepackage{amsmath}

% Title Page
\title{Projekt z eksploracji danych}
\author{Filip Chudy, Marcin Grzywaczewski}


\begin{document}
\maketitle

\begin{abstract}

\end{abstract}
\section{Opis danych i badanego zagadnienia}
\subsection{Dane}
W projekcie wykorzystany został Million Songs Dataset (Last.fm). 
Składa się on z informacji na temat miliona piosenek.
Każda piosenka reprezentowana jest jako JSON z następującymi elementami:
\begin{itemize}
 \item track\_id -- string, unikalny identyfikator piosenki
 \item similars -- lista par (track\_id, float) określająca podobieństwa między piosenkami wraz z ich ważnością (0-1)
 \item tags -- lista par (string, float) określająca tagi przypisane utworowi wraz z ich ważnością (0-100)
 \item artist -- string, nazwa artysty
 \item title -- string, tytuł piosenki
 \item timestamp -- string, czas zebrania danych
\end{itemize}

Z naszego punktu widzenia pole \textbf{timestamp} jest nieistotne.
Pola \textbf{artist} i \textbf{title} służą jedynie do prezentacji wyników.
Liczba utworów podobnych i podanych tagów jest zmienna, w szczególności mogą się zdarzyć takie utwory, które nie mają podanych tagów.
Liczba piosenek podanych w similars nie jest zbyt duża, co obrazuje histogram.
% TODO histogram

\subsection{Cel}
Celem projektu jest stworzenie systemu rekomendującego działającego na powyższych danych.
Zadanie jest trudne ze względu na rzadkość i niekompletność macierzy tagów i podobieństwa.

\section{Użyte metody}
\subsection{Wstęp}
Dane są rzadkie, więc wektoryzacja piosenek na podstawie tagów nie wchodziła w grę. 
Struktura \textbf{similars} przywodzi na myśl graf, dzięki czemu można uzupełnić macierz podobieństwa choćby za pomocą obliczenia przechodniego domknięcia.
Ponieważ w grafach naturalną miarą jest odległość, przeszliśmy z opisu opartego na podobieństwach na odległości według wzoru
\begin{equation}
 distance(x, y) = \frac{1}{similarity(x, y)}
\end{equation}

Metody wyznaczania przechodniego domknięcia grafu mają złożoność sześcienną względem liczby wierzchołków w grafie, 
więc aby którąś zastosować, należy zmniejszyć graf.

Pomysł na redukcję grafu wywodzi się z grupowania hierarchicznego.
Gęste skupiska punktów można reprezentować przy pomocy pojedynczych superwierzchołków, w których rozwiązywany będzie podobny problem dla tego skupiska.
Na tak wyznaczonych grafach hierarchicznych można już wyznaczyć przechodnie domknięcia.

\subsection{Algorytm hierarchiczny}
Algorytm kompresujący graf jest bardzo podobny w działaniu do algorytmów grupowania hierarchicznego.
Dane będą reprezentowane przez graf. Jego wierzchołki odpowiadają piosenkom lub ich zbiorom, a krawędzie -- odległościom między nimi.
W wierzchołkach grafu będzie przechowywana mniejsza wersja rozwiązywanego problemu.

Graf $G_1 = (V_1, E_1)$ jest przodkiem grafu $G_2 = (V_2, E_2)$, jeżeli istnieje taka krawędź $e(v_i, v_j) \in E_1$, że $V_1 \setminus \{v_i, v_j\} \subseteq V_2$, 
$V_2 \setminus V_1 = \{v_k\}$ i $inner\_vertices(v_k) = inner\_vertices(v_i) \cup inner\_vertices(v_j)$.\linebreak
Dodatkowo $(\forall w \in \{v: e(v_i, v) \in E_1 \vee e(v_j, v) \in E_1\})~e(v_k, w) \in E_2 \wedge |e(v_k, w)| = \min \{|e(u, w)| : u \in \{v_i, v_j\}\}$.
$G_2$ nazywamy potomkiem grafu $G_1$, jeśli $G_1$ jest przodkiem $G_2$.

Innymi słowy dla każdej krawędzi $e$ w grafie $G_1$ istnieje potomek $G_2$ grafu $G_1$ powstały przez scalenie wierzchołków łączonych przez $e$.
Oczywiście każda krawędź w grafie jednoznacznie wyznacza odpowiadającego jej potomka.

Każdemu grafowi przypisujemy koszt według pewnej funkcji błędu $F$.
W każdej iteracji wybieramy pewnego potomka $G_2$ grafu $G_1$ pod warunkiem, że $F(G_2) < F(G_1)$.
Ponieważ wyznaczanie wartości funkcji $F$ dla każdego potomka grafu $G_1$ byłoby kosztowne obliczeniowo, zastosowaliśmy wybór heurystyczny.
Wybieramy najkrótszą taką krawędź $e(v_i, v_j)$, że graf $G_{ij}$ powstały przez scalenie $v_i$ i $v_j$ ma niższą funkcję błędu niż jego przodek.

Wybór kończymy, jeśli każdy potomek grafu ma nie mniejszy koszt niż jego przodek lub przekroczymy minimalną wielkość grafu (parametr algorytmu).

\subsection{Funkcja kosztu}
Funkcja kosztu składa się z trzech czynników: WCD (rozmiary wewnętrzne superwierzchołków), 
SVP (penalizacja superwierzchołków jednoelementowych), BCD (odległości między superwierzchołkami).

\begin{equation}
 WCD(G) = \sum_{c \in C} \max_{e \in E(graph(c)} |e|
\end{equation}
\begin{equation}
 SVP(G) = |\{c \in C : size(c) = 1\}| \times \max_{e \in E(C)} |e|
\end{equation}
\begin{equation}
 BCD(G) = \sum_{c_1 \in C} \sum_{c_2 \in C} |e(c_1, c_2)|
\end{equation}

Sama funkcja kosztu jest iloczynem tych trzech czynników: $F(G) = WCD(G) \times SVP(G) \times BCD(G)$.

Obliczanie takiej funkcji jest kosztowne, jednak jeśli znamy wartości wszystkich czynników składających się na funkcję kosztu przodka, 
możemy na ich podstawie efektywnie wyznaczyć czynniki dla potomka.

Niech $G_2$ będzie potomkiem $G_1$ łączącym $c_i, c_j \in G_1$ w $c_k \in G_2$. Można zauważyć, że:
\begin{equation}
\begin{split}
 \Delta WCD(G_2) = WCD(G_2) - WCD(G_1)\\
 = \sum_{c \in C(G_2)} \max_{e \in E(graph(c))} |e| - \sum_{c \in C(G_1)} \max_{e \in E(graph(c))} |e|\\
 = \max_{e \in E(graph(v_k))} |e| - \max_{e \in E(graph(v_i))} |e| - \max_{e \in E(graph(v_j))} |e|
 \end{split}
\end{equation}

\begin{equation}
\begin{split}
 \Delta SVP(G_2) = SVP(G_2) - SVP(G_1)\\
 = |\{c \in \{v_i, v_j\} : size(c) = 1\}| \times \max_{e \in E(G)} |e|
 \end{split}
\end{equation}

\begin{equation}
\begin{split}
 \Delta BCD(G_2) = BCD(G_2) - BCD(G_1)\\
 = \sum_{c_1 \in C_2} \sum_{c_2 \in C_2} |e(c_1, c_2)| - \sum_{c_1 \in C_1} \sum_{c_2 \in C_1} |e(c_1, c_2)|\\
 = \sum_{c \in C_2} |e(v_k, c)| - \sum_{c \in C_1} |e(v_i, c)| - \sum_{c \in C_1} |e(v_j, c)| + |e(v_i, v_j)|
 \end{split}
\end{equation}

Dzięki temu obliczanie funkcji kosztu dla kolejnych potomków grafu jest znacznie szybsze.

\section{Implementacja}

TODO liczenie iteracji jak powyżej bolało, więc ucinamy listę znacznie bardziej chamsko
\section{Wyniki}
\section{Wnioski końcowe, podsumowanie, perspektywy rozwoju}
\end{document}          
