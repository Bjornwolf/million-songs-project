\documentclass[a4paper,10pt]{article}
\usepackage[utf8]{inputenc}
\usepackage{polski}
\usepackage[polish]{babel}

% Title Page
\title{Projekt z eksploracji danych}
\author{Filip Chudy, Marcin Grzywaczewski}


\begin{document}
\maketitle

\begin{abstract}

\end{abstract}
\section{Opis danych i badanego zagadnienia}
\subsection{Dane}
W projekcie wykorzystany został Million Songs Dataset (Last.fm). 
Składa się on z informacji na temat miliona piosenek.
Każda piosenka reprezentowana jest jako JSON z następującymi elementami:
\begin{itemize}
 \item track\_id -- string, unikalny identyfikator piosenki
 \item similars -- lista par (track\_id, float) określająca podobieństwa między piosenkami wraz z ich ważnością (0-1)
 \item tags -- lista par (string, float) określająca tagi przypisane utworowi wraz z ich ważnością (0-100)
 \item artist -- string, nazwa artysty
 \item title -- string, tytuł piosenki
 \item timestamp -- string, czas zebrania danych
\end{itemize}

Z naszego punktu widzenia pole \textbf{timestamp} jest nieistotne.
Pola \textbf{artist} i \textbf{title} służą jedynie do prezentacji wyników.
Liczba utworów podobnych i podanych tagów jest zmienna, w szczególności mogą się zdarzyć takie utwory, które nie mają podanych tagów.
Liczba piosenek podanych w similars nie jest zbyt duża, co obrazuje histogram.
% TODO histogram

\subsection{Cel}
Celem projektu jest stworzenie systemu rekomendującego działającego na powyższych danych.
Zadanie jest trudne ze względu na rzadkość i niekompletność macierzy tagów i podobieństwa.

\section{Użyte metody}
Dane są rzadkie, więc wektoryzacja piosenek na podstawie tagów nie wchodziła w grę. 
Struktura \textbf{similars} przywodzi na myśl graf, dzięki czemu można uzupełnić macierz podobieństwa choćby za pomocą obliczenia przechodniego domknięcia.
Ponieważ w grafach naturalną miarą jest odległość, przeszliśmy z opisu opartego na podobieństwach na odległości według wzoru
\begin{equation}
 distance(x, y) = \frac{1}{similarity(x, y)}
\end{equation}

Metody wyznaczania przechodniego domknięcia grafu mają złożoność sześcienną względem liczby wierzchołków w grafie, 
więc aby którąś zastosować, należy zmniejszyć graf.

Pomysł na redukcję grafu wywodzi się z grupowania hierarchicznego.
Gęste skupiska punktów można reprezentować przy pomocy pojedynczych superwierzchołków, w których rozwiązywany będzie podobny problem dla tego skupiska.
Na tak wyznaczonych grafach hierarchicznych można już wyznaczyć przechodnie domknięcia.

\section{Implementacja}
\section{Wyniki}
\section{Wnioski końcowe, podsumowanie, perspektywy rozwoju}
\end{document}          
